%%%%%%%%%%%%%%%%%%%%%%%%%%%%%%%%%%%%%%%%%%%%%%%%%%%%%%%%%%%%%%%%%%%%%%
%
% レポートテンプレート
%
% updated 22 Oct, 2018
% last updated 02 Apr, 2021
%
% (c) Tohru TAKADA@UEC
% 各自のレポートに合わせて変更して使ってください.上記の行は残して使うこと.
% 2次配布可です.ご利用は計画的に.
%
%%%%%%%%%%%%%%%%%%%%%%%%%%%%%%%%%%%%%%%%%%%%%%%%%%%%%%%%%%%%%%%%%%%%%%
\documentclass[a4paper,10pt]{jarticle}
\usepackage[dvipdfmx]{graphicx}
\usepackage{amsmath}
\usepackage{latexsym}
\usepackage{multirow}
\usepackage{url}
\setlength{\textwidth}{165mm} %165mm-marginparwidth
\setlength{\marginparwidth}{40mm}
\setlength{\textheight}{225mm}
\setlength{\topmargin}{-5mm}
\setlength{\oddsidemargin}{-3.5mm}

\def\vector#1{\mbox{\boldmath $#1$}}
\newcommand{\AmSLaTeX}{%
 $\mathcal A$\lower.4ex\hbox{$\!\mathcal M\!$}$\mathcal S$-\LaTeX}
\newcommand{\PS}{{\scshape Post\-Script}}
\def\BibTeX{{\rmfamily B\kern-.05em{\scshape i\kern-.025em b}\kern-.08em
 T\kern-.1667em\lower.7ex\hbox{E}\kern-.125em X}}
\newcommand{\pderiv}[2]{{\partial#1\over\partial#2}}
\newcommand{\deriv}[2]{{{\rm d}#1\over{\rm d}#2}}
\newcommand{\dderiv}[2]{{{\rm d}^2#1\over{\rm d}#2^2}}
\newcommand{\DeLta}{{\mit\Delta}}
\renewcommand{\d}{{\rm d}}
\def\wcaption#1{\caption[]{\parbox[t]{100mm}{#1}}}
\def\rm#1{\mathrm{#1}}
\def\tempC{^\circ \rm{C}}

\makeatletter
%\def\section{\@startsection {section}{1}{\z@}{-3.5ex plus -1ex minus % -.2ex}{2.3ex plus .2ex}{\Large\bf}}
\def\section{\@startsection {section}{1}{\z@}{-3.5ex plus -1ex minus
-.2ex}{2.3ex plus .2ex}{\normalsize\bf}}
\makeatother

\makeatletter
\def\subsection{\@startsection {subsection}{1}{\z@}{-3.5ex plus -1ex minus
-.2ex}{2.3ex plus .2ex}{\normalsize\bf}}
\makeatother

\makeatletter
\def\@seccntformat#1{\@ifundefined{#1@cntformat}%
   {\csname the#1\endcsname\quad}%      default
   {\csname #1@cntformat\endcsname}%    enable individual control
}
\makeatother

%%%%%%%%%%%%%%%%%%%%%%%%%%%%%%%%%%%%%%%%%%%%%%%%%%%%%%%%%%%%%%%%%%%%%%
\begin{document}

%
%% 通常は指定の表紙を付けて印刷して提出していますが,電子データをアップロードする際は
%% 表紙は付けなくても結構です.冒頭にタイトル,所属,学籍番号,氏名と記載日
%% 修正版を提出する際は更新日を書いてください
%
\begin{center}
{\Large{\bf K演習第6回レポート課題}} \\
{\bf Aクラス 2311009 アハメドアティフ} \\

\end{center}
%%%%%%%%%%%%%%%%%%%%%%%%%%%%%%%%%%%%%%%%%%%%%%%%%%%%%%%%%%%%%%%%%%%%%%
\section{目的}


\vspace{3mm}


%%%%%%%%%%%%%%%%%%%%%%%%%%%%%%%%%%%%%%%%%%%%%%%%%%%%%%%%%%%%%%%%%%%%%%
\section{原理}

\begin{equation}
T=2\pi \sqrt{\frac{h}{g}\left (1+\frac{2r^2}{5h^2}\right )} \times \left (1+\frac{\theta^2}{16}\right )
\label{grav_eq}
\end{equation}

実験の原理では上式で原理を終えるのは良い方法ではありません.なぜなら,あなたが
求めるべき量は重力加速ですから,式(\ref{grav_eq})を $g$ について解いた
ものを載せるのが良い方法です.
%
\begin{equation}
g=\frac{4\pi^2 h}{T^2}\times \left (1+\frac{2r^2}{5h^2} +\frac{\theta^2}{8}\right )
\label{grav2_eq}
\end{equation}

一行独立に書いている式には必ず式番号を割り振ってください.式のインデントの付け方は
分野によって幾つか流儀があるようです.物理では式を中央寄せ,数式番号は右寄せに
して行をそろえるのが一般的です.

\vspace{3mm}
\TeX では数式にはラベル(\verb|\label{<label>}|)を付けておき,後で参照(\verb|\ref{<label>}|)すると
式番号を自動的に引用します(後で途中に式を追加してもいちいち付け直す必要はありません).
ただし,引用は一度のコンパイルでは解決できない場合があります(??のように式番号が未定に
なります.メッセージを読むと警告が表示されているはずです).このようなときは
もう一度コンパイルをしてください.また,式をコピペしているときに起こりがちですが,ラベルは重複して
用いることはできません.ペーストした際はラベルを書き換えることを忘れないでください.

数式を書くのは \TeX に慣れるまでは存外面倒なものです.各課題について代表的な数式を
ここで掲載しておきましょう.どのように数式コマンドを書けばよいかの参考にしてください.

%
\subsection{重力加速度の測定}
不確かさを求める式は次のようになります.
\begin{eqnarray}
\overline{g}=\cfrac{\cfrac{g_1}{(\Delta g_1)^2}+\cfrac{g_2}{(\Delta g_2)^2}+\cdots+\cfrac{g_n}{(\Delta g_n)^2}}{\cfrac{1}{(\Delta g_1)^2}+\cfrac{1}{(\Delta g_2)^2}+\cdots+\cfrac{1}{(\Delta g_n)^2}} \\
\label{eq1}
\Delta g = \cfrac{1}{\sqrt{\cfrac{1}{(\Delta g_1)^2}+\cfrac{1}{(\Delta g_2)^2}+\cdots+\cfrac{1}{(\Delta g_n)^2}}}
\label{eq2}
\end{eqnarray}

%
\subsection{音の共鳴}
固体中の波の速さ $v_m$ に対する合成標準不確かさを求める式.
\begin{equation}
\frac{\Delta v_m}{v_m}=\sqrt{\left (\frac{\Delta L_g}{L_g}\right )^2+\left (\frac{\Delta l_m}{l_m}\right )^2+\left (\frac{\Delta v_g}{v_g}\right )^2}
\label{eq3}
\end{equation}

\subsection{液体の比熱}
加熱法による液体試料の温度上昇と時間の関係を表す式.
\begin{equation}
\frac{\Delta T}{\Delta t}=\frac{Ri^2}{MC+mc}
\label{eq4}
\end{equation}

\subsection{2次元の等電位線}
中央に導体がある等電位線の理論式の作図に関する式.
\begin{equation}
x=\pm \sqrt{\frac{y(-y^2+cy+R^2)}{y-c}}
\label{eq5}
\end{equation}

\subsection{電気回路}
抵抗,コイル,コンデンサの直列回路による過渡応答の原理式.
\begin{equation}
\dderiv{I(t)}{t}+2\gamma \deriv{I(t)}{t}+\omega_0^2I(t)=0
\label{eq6}
\end{equation}

ここで用いている\verb|\ddereiv|や\verb|\deriv|はこのファイルの冒頭に定義してあるもので,通常の
\TeX のコマンドではないことに注意が必要です.

\subsection{ヤング率}
たわみによるヤング率を求める式.
\begin{equation}
E=\frac{g}{2}\frac{l^3 d}{a^3 br}\frac{m}{S-S_0}
\label{eq7}
\end{equation}

\subsection{粘性率}
ポアズイユの法則を表す式.
\begin{equation}
p_1-p_2=\left (\rho g\cos\theta -\frac{2\gamma}{al}\right )l
\label{eq8}
\end{equation}

\subsection{光のスペクトル}
Na線の D$_1$線とD$_2$線の波長.

 D$_1$線:589.592 nm \\
 
 ${\rm D_2}$線:588.995 $\rm{nm}$

\subsection{エアトラックによる力学実験}
滑走体のエアトラック上の運動方程式を減速の平均の加速度 $\overline{a}$ と平均速度 $\overline{v}$ で記述した式.
\begin{equation}
\frac{\overline{a}}{g}=\mu \cos\theta +\left (\frac{\lambda}{mg}\right )\overline{v}+\left (\frac{\kappa}{mg}\right )\overline{v}^2\mp\sin\theta
\label{eq9}
\end{equation}


\begin{equation}
^{137}_{\hspace{1.2mm}55}\rm{Cs} \rightarrow ^{137}_{\hspace{1.2mm}56}\rm{Ba} + \rm{e}^{-} +\overline{\nu}_{\rm{e}}
\label{beta-colps}
\end{equation}

%%%%%%%%%%%%%%%%%%%%%%%%%%%%%%%%%%%%%%%%%%%%%%%%%%%%%%%%%%%%%%%%%%%%%%
\section{方法}

%
\begin{itemize}
\setlength{\itemsep}{-2mm}
 \item 測定上特に注意をした点(注意をしなければならなかった点)
 \item 測定装置で特に説明をしなければ何故,どのように測定値を取得したのか
 	  読者にわからない点
 \item 実験結果を再現するために必要な特別な手順や測定方法など
 \item 測定データを処理する際に利用したソフトウェアで特に記載が必要なもの
\end{itemize}

\begin{enumerate}
\setlength{\itemsep}{-2mm}
 \item 電圧はディジタルマルチメーターを用いて測った
 \item まずディジタルマルチメーターの電源をONにした
 \item 次にメーターのプローブを電源端子に接続し$\cdots$
\end{enumerate}



%%%%%%%%%%%%%%%%%%%%%%%%%%%%%%%%%%%%%%%%%%%%%%%%%%%%%%%%%%%%%%%%%%%%%%
\section{実験結果}

 実験結果のセクションでは得られた測定値を元にして目的のセクションで掲げた物理量を
計算し,(存在する場合は)比較できる文献値等と一緒に最後に表にしてまとめます.もちろん,
不確かさや精度を吟味できる場合,同様のこのセクションで計算を行って求めた値と合わせて示します.

データをまとめる際は必ず表にしてまとめてください.またはデータを
グラフにプロットして表示することは良いアイディアで,多くの場合傾向が大変分かりやすくなります.
表やグラフをどのように書くのかについてはeラーニングの課題として与えています.よく習熟して
ください.表のサンプルは次の表\ref{tab1}のようなものです.
%
\begin{table}[ht]
\begin{center}
\caption{ooにおける回折角の測定結果}
\label[tab1]
\begin{tabular}{lllll}\hline
 & \multicolumn{2}{c}{1次回折光} & \multicolumn{2}{c}{2次回折光} \\
 & D$_1$線 &  D$_2$線 & D$_1$線 & D$_2$線 \\ \hline
$\theta_{\rm L}$ &  287$^\circ$35' & 287$^\circ$37' & 311$^\circ$37' & 311$^\circ$33' \\
$\theta_{\rm L}$ &  246$^\circ$9' & 221$^\circ$10' & 221$^\circ$32' & 221$^\circ$33' \\ \hline
\end{tabular}
\end{center}
\end{table}


%%%%%%%%%%%%%%%%%%%%%%%%%%%%%%%%%%%%%%%%%%%%%%%%%%%%%%%%%%%%%%%%%%%%%%


%%%%%%%%%%%%%%%%%%%%%%%%%%%%%%%%%%%%%%%%%%%%%%%%%%%%%%%%%%%%%%%%%%%%%%
\end{document}

\begin{figure}[ht]
\begin{center}
 \includegraphics[scale=0.5]{ファイル名}
 \caption{タイトル}
 %\ecaption{Options of documentclass.}
 \label{apara}
\end{center}
\end{figure}

\begin{table}[ht]
\begin{center}
\caption{タイトル}
\label{tab1}
\begin{tabular}{ll}\hline
col1 & col2 \\ \hline
val1 & val2 \\
val3 & val4 \\ \hline
\end{tabular}%
\end{center}
\end{table}
