%%%%%%%%%%%%%%%%%%%%%%%%%%%%%%%%%%%%%%%%%%%%%%%%%%%%%%%%%%%%%%%%%%%%%%
%
% レポートテンプレート
%
% updated 22 Oct, 2018
% last updated 02 Apr, 2021
%
% (c) Tohru TAKADA@UEC
% 各自のレポートに合わせて変更して使ってください.上記の行は残して使うこと.
% 2次配布可です.ご利用は計画的に.
%
%%%%%%%%%%%%%%%%%%%%%%%%%%%%%%%%%%%%%%%%%%%%%%%%%%%%%%%%%%%%%%%%%%%%%%
\documentclass[a4paper,10pt]{jarticle}
\usepackage[dvipdfmx]{graphicx}
\usepackage{amsmath}
\usepackage{latexsym}
\usepackage{multirow}
\usepackage{url}
\usepackage{float}
\setlength{\textwidth}{165mm} %165mm-marginparwidth
\setlength{\marginparwidth}{40mm}
\setlength{\textheight}{225mm}
\setlength{\topmargin}{-5mm}
\setlength{\oddsidemargin}{-3.5mm}

\def\vector#1{\mbox{\boldmath $#1$}}
\newcommand{\AmSLaTeX}{%
 $\mathcal A$\lower.4ex\hbox{$\!\mathcal M\!$}$\mathcal S$-\LaTeX}
\newcommand{\PS}{{\scshape Post\-Script}}
\def\BibTeX{{\rmfamily B\kern-.05em{\scshape i\kern-.025em b}\kern-.08em
 T\kern-.1667em\lower.7ex\hbox{E}\kern-.125em X}}
\newcommand{\pderiv}[2]{{\partial#1\over\partial#2}}
\newcommand{\deriv}[2]{{{\rm d}#1\over{\rm d}#2}}
\newcommand{\dderiv}[2]{{{\rm d}^2#1\over{\rm d}#2^2}}
\newcommand{\DeLta}{{\mit\Delta}}
\renewcommand{\d}{{\rm d}}
\def\wcaption#1{\caption[]{\parbox[t]{100mm}{#1}}}
\def\rm#1{\mathrm{#1}}
\def\tempC{^\circ \rm{C}}

\makeatletter
%\def\section{\@startsection {section}{1}{\z@}{-3.5ex plus -1ex minus % -.2ex}{2.3ex plus .2ex}{\Large\bf}}
\def\section{\@startsection {section}{1}{\z@}{-3.5ex plus -1ex minus
-.2ex}{2.3ex plus .2ex}{\normalsize\bf}}
\makeatother

\makeatletter
\def\subsection{\@startsection {subsection}{1}{\z@}{-3.5ex plus -1ex minus
-.2ex}{2.3ex plus .2ex}{\normalsize\bf}}
\makeatother

\makeatletter
\def\@seccntformat#1{\@ifundefined{#1@cntformat}%
   {\csname the#1\endcsname\quad}%      default
   {\csname #1@cntformat\endcsname}%    enable individual control
}
\makeatother

%%%%%%%%%%%%%%%%%%%%%%%%%%%%%%%%%%%%%%%%%%%%%%%%%%%%%%%%%%%%%%%%%%%%%%
\begin{document}

%
%% 通常は指定の表紙を付けて印刷して提出していますが,電子データをアップロードする際は
%% 表紙は付けなくても結構です.冒頭にタイトル,所属,学籍番号,氏名と記載日
%% 修正版を提出する際は更新日を書いてください
%
\begin{center}
{\Large{\bf K演習第14回レポート課題}} \\
{\bf Aクラス 2311009 アハメドアティフ} \\

\end{center}
%%%%%%%%%%%%%%%%%%%%%%%%%%%%%%%%%%%%%%%%%%%%%%%%%%%%%%%%%%%%%%%%%%%%%%
\section{宿題2-1}

\begin{enumerate}
	\setlength{\itemsep}{-2mm}
	 \item 標本空間は0以上の整数で, $\Omega = \{\ 0,1,2\cdots\}\ $,母数空間は正の実数で,$\{\lambda|0<\lambda <\infty\}\ $である.

  \vspace{6mm}
%%%%%%%%%%%%%%%%%%%%%%%%%%%%%%%%%%%%%%%%%%%%%%%%%%%%%%%%%%%%%%%%%%%%%%


	 \item ポアソン分布の平均は期待値を求めることで以下の式ように求められた.

	 \begin{equation}
		\label{equ1}
		\begin{split}
		E[X] &= \sum_{x = 0}^{\infty}x\frac{{\lambda}^x e^{-\lambda}}{x!} = \lambda\sum_{x = 1}^{\infty}\frac{{\lambda}^{x-1} e^{-\lambda}}{(x-1)!} = \lambda\sum_{y = 0}^{\infty}\frac{{\lambda}^y e^{-\lambda}}{y!} = \lambda
		\end{split}
	\end{equation}

	以上の式では,$y=x-1$という置換と,$\sum_{y = 0}^{\infty}\frac{{\lambda}^y e^{-\lambda}}{y!}$がポアソン分布の全確率の和を表していることから$\sum_{y = 0}^{\infty}\frac{{\lambda}^y e^{-\lambda}}{y!}= 1$であることを利用している.したがって,求める平均値は\fbox{$\lambda$}である.

	\vspace{6mm}
	%%%%%%%%%%%%%%%%%%%%%%%%%%%%%%%%%%%%%%%%%%%%%%%%%%%%%%%%%%%%%%%%%%%%%%


	 \item まず$E[X(X-1)] $を求めると以下のようになった.
	 
	 \begin{equation}
		\label{equ2}
		\begin{split}
		E[X] &= \sum_{x = 0}^{\infty}x(x-1)\frac{{\lambda}^x e^{-\lambda}}{x!} = {\lambda}^2\sum_{x = 2}^{\infty}\frac{{\lambda}^{x-2} e^{-\lambda}}{(x-2)!} = {\lambda}^2\sum_{y = 0}^{\infty}\frac{{\lambda}^y e^{-\lambda}}{y!} = {\lambda}^2
		\end{split}
	\end{equation}

	以上の式では,$y=x-2$という置換と,2と同様に$\sum_{y = 0}^{\infty}\frac{{\lambda}^y e^{-\lambda}}{y!}$がポアソン分布の全確率の和を表していることから$\sum_{y = 0}^{\infty}\frac{{\lambda}^y e^{-\lambda}}{y!}= 1$であることを利用している.したがって,求める分散の値は以下のように求められた.

	\begin{equation}
		\label{equ3}
		\begin{split}
		V[X] = E[X(X-1)] + E[X] +- E[X^2] = {\lambda}^2 + \lambda - {\lambda}^2 = \lambda
		\end{split}
	\end{equation}

よって,求める分散の値は\fbox{${\lambda}^2$}.

	 \vspace{6mm}
%%%%%%%%%%%%%%%%%%%%%%%%%%%%%%%%%%%%%%%%%%%%%%%%%%%%%%%%%%%%%%%%%%%%%%


	 \item ポアソン分布のモーメント母関数は以下のように求めた.
	 
	 \begin{equation}
		\label{equ4}
		\begin{split}
		M(t) = E[e^{tX}] = \sum_{x = 0}^{\infty}e^{tx}\frac{{\lambda}^x e^{-\lambda}}{x!} = \sum_{x = 0}^{\infty}\frac{{(\lambda e^t)}^x e^{-\lambda}}{x!} 
		\end{split}
	\end{equation}
	 
	ここで,$\sum_{k = 0}^{\infty}\frac{x^k}{k!} = e^x$という性質から,

	\begin{equation}
		\label{equ5}
		\begin{split}
		M(t) = \sum_{x = 0}^{\infty}\frac{{(\lambda e^t)}^x e^{-\lambda}}{x!} = e^{\lambda e^t}e^{-\lambda} = e^{\lambda(e^t-1)}
		\end{split}
	\end{equation}

	 したがって,求めるモーメント母関数は\fbox{$e^{\lambda(e^t-1)}$}である.

	 \vspace{6mm}
%%%%%%%%%%%%%%%%%%%%%%%%%%%%%%%%%%%%%%%%%%%%%%%%%%%%%%%%%%%%%%%%%%%%%%


	 \item  4で求めたモーメント母関数をtで一回微分したものに$t=0$を代入することによって$X$の期待値を得ることができる.まず$M(t)$を一回微分すると,以下のようになる.
	 
	 \begin{equation}
		\label{equ6}
		\begin{split}
		M'(t) = \lambda e^t e^{\lambda(e^t-1)}
		\end{split}
	\end{equation}

	よって,ここに$t=0$を代入すると,

	\begin{equation}
		\label{equ7}
		\begin{split}
		E[X] = M'(0) = \lambda e^0 e^{\lambda(e^0-1)} = \lambda \cdot 1 \cdot 1 = \lambda
		\end{split}
	\end{equation}
	
  と以上のようになり,求める$X$の期待値$E[X]$は\fbox{$\lambda$}である.

	次に,$X^2$の期待値は,$M'(t)$をさらに微分して,$t=0$を代入することによって得られる.まず微分をすると,

	\begin{equation}
		\label{equ8}
		\begin{split}
		M''(t) = (\lambda e^{\lambda(e^t-1)+t})' = \lambda(\lambda e^t +1)e^{\lambda(e^t-1)+t}
		\end{split}
	\end{equation}

	よって,$t=0$を代入すると,$X^2$の期待値は,

	\begin{equation}
		\label{equ9}
		\begin{split}
		E[X^2] = M''(0) = \lambda(\lambda e^0 +1)e^{\lambda(e^0-1)+0} = \lambda (\lambda \cdot 1 + 1) \cdot 1 = \lambda ^2 + \lambda
		\end{split}
	\end{equation}

と以上のようになり,$X^2$の期待値は\fbox{$\lambda ^2 + \lambda$}である.

	 \vspace{6mm}
%%%%%%%%%%%%%%%%%%%%%%%%%%%%%%%%%%%%%%%%%%%%%%%%%%%%%%%%%%%%%%%%%%%%%%


	 \item  $X_1$と$X_2$が互いにポアソン分布$Pois(\lambda_1)$と$Pois(\lambda_2)$に従うとき,その$X_1$と$X_2$の和の確率変数$Y$のモーメント母関数は以下のように変形できる.
	 
	 \begin{equation}
		\label{equ10}
		\begin{split}
		M_Y(t) = M_{X_1}(t)M_{X_2}(t) = e^{\lambda_1(e^t-1)}e^{\lambda_2(e^t-1)} = e^{(\lambda_1 + \lambda_2)(e^t-1)}
		\end{split}
	\end{equation}
	
よって,これはポアソン分布の表現に整理できているので,$Y$もポアソン分布に従っていて,その確率関数は

\begin{equation}
	\label{equ11}
	\begin{split}
	p(y;\lambda_1 + \lambda_2) = \frac{(\lambda_1 + \lambda_2)^y e^{-(\lambda_1 + \lambda_2)}}{y!}
	\end{split}
\end{equation}

と表されるので,\fbox{$p(y;\lambda_1 + \lambda_2) = \frac{(\lambda_1 + \lambda_2)^y e^{-(\lambda_1 + \lambda_2)}}{y!}$}.
	\end{enumerate}


%%%%%%%%%%%%%%%%%%%%%%%%%%%%%%%%%%%%%%%%%%%%%%%%%%%%%%%%%%%%%%%%%%%%%%
%%%%%%%%%%%%%%%%%%%%%%%%%%%%%%%%%%%%%%%%%%%%%%%%%%%%%%%%%%%%%%%%%%%%%%
%%%%%%%%%%%%%%%%%%%%%%%%%%%%%%%%%%%%%%%%%%%%%%%%%%%%%%%%%%%%%%%%%%%%%%

\section{宿題2-2}

\begin{enumerate}
	\setlength{\itemsep}{-2mm}
	 
	\item 正規分布の定義から,標本空間は $\Omega = \{\ \infty < x < x \}\ $,母数空間は$\{\mu|-\infty <\mu <\infty\}\ $,$\{\sigma^2|0 <\sigma^2 <\infty\}\ である.$

\vspace{6mm}
%%%%%%%%%%%%%%%%%%%%%%%%%%%%%%%%%%%%%%%%%%%%%%%%%%%%%%%%%%%%%%%%%%%%%%


\item 正規分布の平均値を求めるにあたって,最初に$E[X-\mu]$について考えていくと,

\begin{equation}
\label{equ12}
\begin{split}
E[X-\mu] &= \int_{-\infty}^{\infty} (x-\mu) \frac{1}{\sqrt{2\pi\sigma^2}} exp\left\{ -\frac{(x-\mu)^2}{2\sigma^2}\right\}  \,dx
\end{split}
\end{equation}

ここで,以上の式の被積分関数は奇関数であり,$x=\mu$に関して対称な関数であるから,

\begin{equation}
	\label{equ13}
	\begin{split}
	\int_{-\infty}^{\mu} (x-\mu) \frac{1}{\sqrt{2\pi\sigma^2}} exp\left\{ -\frac{(x-\mu)^2}{2\sigma^2}\right\}  \,dx = -\int_{\mu}^{\infty} (x-\mu) \frac{1}{\sqrt{2\pi\sigma^2}} exp\left\{ -\frac{(x-\mu)^2}{2\sigma^2}\right\}  \,dx
	\end{split}
	\end{equation}

のような関係が導ける.

\vspace{6mm}
%%%%%%%%%%%%%%%%%%%%%%%%%%%%%%%%%%%%%%%%%%%%%%%%%%%%%%%%%%%%%%%%%%%%%%


これより,$E[X-\mu]$は,

\begin{equation}
	\label{equ14}
	\begin{split}
	E[X-\mu] &= \int_{-\infty}^{\infty} (x-\mu) \frac{1}{\sqrt{2\pi\sigma^2}} exp\left\{ -\frac{(x-\mu)^2}{2\sigma^2}\right\}  \,dx \\
					 &= \int_{-\infty}^{\mu} (x-\mu) \frac{1}{\sqrt{2\pi\sigma^2}} exp\left\{ -\frac{(x-\mu)^2}{2\sigma^2}\right\}  \,dx + \int_{\mu}^{\infty} (x-\mu) \frac{1}{\sqrt{2\pi\sigma^2}} exp\left\{ -\frac{(x-\mu)^2}{2\sigma^2}\right\}  \,dx \\
					 &= \int_{-\infty}^{\mu} (x-\mu) \frac{1}{\sqrt{2\pi\sigma^2}} exp\left\{ -\frac{(x-\mu)^2}{2\sigma^2}\right\}  \,dx - \int_{-\infty}^{\mu} (x-\mu) \frac{1}{\sqrt{2\pi\sigma^2}} exp\left\{ -\frac{(x-\mu)^2}{2\sigma^2}\right\}  \,dx \\
					 &= 0
	\end{split}
	\end{equation}

となる.よって,線形性から

\begin{equation}
	\label{equ15}
	\begin{split}
	E[X-\mu] = 0 \Longleftrightarrow E[X] = \mu
	\end{split}
	\end{equation}

が得られ,求める平均値は\fbox{$\mu$}となる.

\vspace{6mm}
%%%%%%%%%%%%%%%%%%%%%%%%%%%%%%%%%%%%%%%%%%%%%%%%%%%%%%%%%%%%%%%%%%%%%%


\item 分散を求めるにあたり,置換積分を用いて解いていきたいので,$E[\frac{(X-\mu)^2}{\sigma^2}]$を考えると,

\begin{equation}
	\label{equ16}
	\begin{split}
	E[\frac{(X-\mu)^2}{\sigma^2}] &= \int_{-\infty}^{\infty} \frac{(x-\mu)^2}{\sigma^2} \frac{1}{\sqrt{2\pi\sigma^2}} exp\left\{ -\frac{(x-\mu)^2}{2\sigma^2}\right\}  \,dx 
	\end{split}
	\end{equation}

ここで,

\begin{equation}
	\label{equ17}
	\begin{split}
	y = \frac{(X-\mu)^2}{\sigma^2} \Longleftrightarrow x = \mu \pm \sigma\sqrt{2y} , dx = \left\lvert \frac{\sigma}{\sqrt{2y}} \right\rvert 
	\end{split}
	\end{equation}

という変数変換と,

\begin{equation}
	\label{equ18}
	\begin{split}
	\int_{0}^{\infty} y^{\frac{3}{2}-1} e^{-y} \,dy = \Gamma\left( \frac{3}{2}\right) = \frac{1}{2}\Gamma\left( \frac{1}{2}\right) = \frac{\sqrt{\pi}}{2}
	\end{split}
	\end{equation}

というガンマ関数の性質,そして被積分関数が偶関数であることから,

\begin{equation}
	\label{equ19}
	\begin{split}
	E[\frac{(X-\mu)^2}{\sigma^2}] &= \int_{-\infty}^{\infty} \frac{(x-\mu)^2}{\sigma^2} \frac{1}{\sqrt{2\pi\sigma^2}} exp\left\{ -\frac{(x-\mu)^2}{2\sigma^2}\right\}  \,dx \\
																&= 2\int_{0}^{\infty} 2y\frac{1}{\sqrt{2\pi\sigma^2}} exp(-y) \frac{\sigma}{\sqrt{2y}}  \,dy \\
																&= 2\int_{0}^{\infty} \frac{\sqrt{y}}{\sqrt{\pi}} exp(-y) \,dy \\
																&= 2\cdot\frac{1}{\sqrt{\pi}}\cdot\frac{\sqrt{\pi}}{2} \\
																&= 1
	\end{split}
	\end{equation}

したがって,分散は

\begin{equation}
	\label{equ20}
	\begin{split}
	V[X] = E[(X-\mu)^2] = \sigma^2
	\end{split}
	\end{equation}

より,\fbox{$\sigma^2$}である.

\vspace{6mm}
%%%%%%%%%%%%%%%%%%%%%%%%%%%%%%%%%%%%%%%%%%%%%%%%%%%%%%%%%%%%%%%%%%%%%%

\item 正規分布のモーメント母関数は以下のように求めた.

\begin{equation}
\label{equ21}
\begin{split}
M(t) &= E[e^{tX}] = \int_{-\infty}^{\infty} e^{tx} \frac{1}{\sqrt{2\pi\sigma^2}} exp\left(\ -\frac{(x-\mu)^2}{2\sigma^2}\right)\ \,dx \\
     &= \int_{-\infty}^{\infty} \frac{1}{\sqrt{2\pi\sigma^2}} exp\left(\ -\frac{(x-\mu)^2}{2\sigma^2} + tx \right)\  \,dx \\
		 &= \int_{-\infty}^{\infty} \frac{1}{\sqrt{2\pi\sigma^2}} exp\left(\ -\frac{x^2 -2\mu x + \mu^2 - 2\sigma^2 xt}{2\sigma^2} \right)\  \,dx \\
		 &= \int_{-\infty}^{\infty} \frac{1}{\sqrt{2\pi\sigma^2}} exp\left(\ -\frac{x^2 -2(\mu+\sigma^2t)x + \mu^2}{2\sigma^2} \right)\  \,dx \\
		 &= \int_{-\infty}^{\infty} \frac{1}{\sqrt{2\pi\sigma^2}} exp\left(\ -\frac{(x-\mu-\sigma^2)^2 -2\mu\sigma^2t - \sigma^4t^2}{2\sigma^2}\right)\  \,dx \\
		 &= exp\left(\ \mu t+ \frac{\sigma^2}{2}t^2\right)\ \int_{-\infty}^{\infty} \frac{1}{\sqrt{2\pi\sigma^2}} exp\left(\ -\frac{(x-\mu-\sigma^2)^2}{2\sigma^2}\right)\  \,dx
		\end{split}
\end{equation}

ここで,$\int_{-\infty}^{\infty} \frac{1}{\sqrt{2\pi\sigma^2}} exp\left(\ -\frac{(x-\mu-\sigma^2)^2}{2\sigma^2}\right)\  \,dx$とは,平均が$\mu+\sigma^2$で分散が$\sigma^2$の正規分布$N(\mu+\sigma^2, \sigma^2)$の全確率を表しているので,$\int_{-\infty}^{\infty} \frac{1}{\sqrt{2\pi\sigma^2}} exp\left(\ -\frac{(x-\mu-\sigma^2)^2}{2\sigma^2}\right)\  \,dx = 1$であるから,

\begin{equation}
	\label{equ22}
	\begin{split}
	M(t) &= exp\left(\ \mu t+ \frac{\sigma^2}{2}t^2\right)\ \int_{-\infty}^{\infty} \frac{1}{\sqrt{2\pi\sigma^2}} exp\left(\ -\frac{(x-\mu-\sigma^2)^2}{2\sigma^2}\right)\  \,dx \\
			 &= exp\left(\ \mu t+ \frac{\sigma^2}{2}t^2\right)\
\end{split}
	\end{equation}

したがって,求めるモーメント母関数は\fbox{$exp\left(\ \mu t+ \frac{\sigma^2}{2}t^2\right)$}である.

\vspace{6mm}
%%%%%%%%%%%%%%%%%%%%%%%%%%%%%%%%%%%%%%%%%%%%%%%%%%%%%%%%%%%%%%%%%%%%%%

\item  4で求めたモーメント母関数をtで一回微分したものに$t=0$を代入することによって$X$の期待値を得ることができる.まず$M(t)$を一回微分すると,以下のようになる.

\begin{equation}
\label{equ23}
\begin{split}
M'(t) = \left(\ \mu + \sigma^2t\right)\exp\left(\ \mu t+ \frac{\sigma^2}{2}t^2\right)\
\end{split}
\end{equation}

よって,ここに$t=0$を代入すると,

\begin{equation} 
\label{equ24}
\begin{split}
E[X] = M'(0) =\left(\ \mu + \sigma^2\cdot 0\right)\exp\left(\ \mu \cdot 0+ \frac{\sigma^2}{2}\cdot 0^2\right)\ = \mu
\end{split}
\end{equation}

のようになり,求める$X$の期待値$E[X]$は\fbox{$\mu$}である.

次に,$X^2$の期待値は,$M'(t)$をさらに微分して,$t=0$を代入することによって得られる.まず微分をすると,

\begin{equation}
\label{equ25}
\begin{split}
M''(t) &= \sigma^2 \exp\left(\ \mu t+ \frac{\sigma^2}{2} t^2\right)\ + \left(\ \mu + \sigma^2t\right)^2 \exp\left(\ \mu t+ \frac{\sigma^2}{2}t^2\right)\ \\
\end{split}
\end{equation}

よって,$t=0$を代入すると,$X^2$の期待値は,

\begin{equation}
\label{equ9}
\begin{split}
E[X^2] &= M''(0) = \sigma^2 \exp\left(\ \mu \cdot 0+ \frac{\sigma^2}{2} \cdot 0^2\right)\ + \left(\ \mu + \sigma^2\cdot 0\right)^2 \exp\left(\ \mu \cdot 0+ \frac{\sigma^2}{2}\cdot 0^2\right)\ \\
			 &= \sigma^2 + \mu^2
\end{split}
\end{equation}

のようになり,$X^2$の期待値は\fbox{$\sigma ^2 + \mu^2$}である.

\vspace{6mm}
%%%%%%%%%%%%%%%%%%%%%%%%%%%%%%%%%%%%%%%%%%%%%%%%%%%%%%%%%%%%%%%%%%%%%%


\item  前問より正規分布$N(\mu,\sigma^2)$のモーメント母関数は$M(t) = exp\left(\ \mu t+ \frac{\sigma^2}{2} t^2\right)\ $であり, $X_1$と$X_2$が互いに独立に正規分布$N(\mu_1,\sigma_1^2)$と$N(\mu_2,\sigma_2^2)$に従うとき,その$X_1$と$X_2$の和の確率変数$Y$のモーメント母関数は以下のように変形できる.

\begin{equation}
\label{equ10}
\begin{split}
M_Y(t) &= M_{X_1}(t)M_{X_2}(t) \\
			 &= exp\left(\ \mu_1 t+ \frac{\sigma_1^2}{2} t^2\right)\ exp\left(\ \mu_2 t+ \frac{\sigma_2^2}{2} t^2\right)\ \\
			 &= exp\left(\ (\mu_1+\mu_2)t+ \frac{\sigma_1^2 + \sigma_2^2}{2} t^2\right)\
\end{split}
\end{equation}

よって,これは平均が$\mu_1+\mu_2$で分散が$\sigma_1^2 + \sigma_2^2$の正規分布の表現に整理できているので,$Y$は正規分布$N(\mu_1+\mu_2,\sigma_1^2 + \sigma_2^2)$に従っていて,その確率密度関数は,

\begin{equation}
\label{equ11}
\begin{split}
f(y;\mu_1+\mu_2,\sigma_1^2 + \sigma_2^2) = \frac{1}{\sqrt{2\pi(\sigma_1^2 + \sigma_2^2)}} exp\left[\ -\frac{ \{\ x-(\mu_1+\mu_2) \}\ ^2}{2(\sigma_1^2 + \sigma_2^2)}\right]\
\end{split}
\end{equation}

と表されるので,\fbox{$\frac{1}{\sqrt{2\pi(\sigma_1^2 + \sigma_2^2)}} exp\left[\ -\frac{ \{\ x-(\mu_1+\mu_2) \}\ ^2}{2(\sigma_1^2 + \sigma_2^2)}\right]\ $}.

	\end{enumerate}
	\vspace{3mm}

%%%%%%%%%%%%%%%%%%%%%%%%%%%%%%%%%%%%%%%%%%%%%%%%%%%%%%%%%%%%%%%%%%%%%%
\end{document}

\begin{figure}[ht]
\begin{center}
 \includegraphics[scale=0.5]{ファイル名}
 \caption{タイトル}
 %\ecaption{Options of documentclass.}
 \label{apara}
\end{center}
\end{figure}

\begin{table}[ht]
\begin{center}
\caption{タイトル}
\label{tab1}
\begin{tabular}{ll}\hline
col1 & col2 \\ \hline
val1 & val2 \\
val3 & val4 \\ \hline
\end{tabular}%
\end{center}
\end{table}

%箇条書き
\begin{itemize}
	\setlength{\itemsep}{-2mm}
	 \item 測定上特に注意をした点(注意をしなければならなかった点)
	 \item 測定装置で特に説明をしなければ何故,どのように測定値を取得したのか
			 読者にわからない点
	 \item 実験結果を再現するために必要な特別な手順や測定方法など
	 \item 測定データを処理する際に利用したソフトウェアで特に記載が必要なもの
	\end{itemize}
	
	%番号振って箇条書き
	\begin{enumerate}
	\setlength{\itemsep}{-2mm}
	 \item 電圧はディジタルマルチメーターを用いて測った
	 \item まずディジタルマルチメーターの電源をONにした
	 \item 次にメーターのプローブを電源端子に接続し$\cdots$
	\end{enumerate}